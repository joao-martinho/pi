\documentclass{article}

\title{Trabalho parcial 02: diagrama de atividades}
\author{Gustavo Guerreiro, João Martinho}
\date{1º de outubro de 2025}

\setlength{\oddsidemargin}{-0.5cm}
\setlength{\evensidemargin}{-0.5cm}
\setlength{\textwidth}{17cm}
\setlength{\topmargin}{-1.5cm}
\setlength{\textheight}{24cm}

\begin{document}

\maketitle

\section{Resumo}

O nosso trabalho usa técnicas clássicas de pré-processamento e segmentação para distinguir e contar os núcleos celulares dos neutrófilos da base de imagens fornecida. A \textit{pipeline} do processamento consiste em:
\begin{itemize}
    \item Isolamento da célula:
    \begin{itemize}
        \item conversão da imagem original para escala de cinza;
        \item limiarização para distinguir a região celular do fundo;
        \item remoção do fundo, mantendo apenas a célula visível.
    \end{itemize}
    \item Segmentação do núcelo:
    \begin{itemize}
        \item aplicação de um novo limiar para isolar o núcleo do citoplasma;
        \item obtenção da imagem binária com os núcleos evidenciados.
    \end{itemize}
\end{itemize}

\section{Tabelas}

Abaixo estão duas tabelas mostrando a quantidade de núcleos encontradas pela nossa solução em cada imagem. A primeira tabela mostra os resultados obtidos com os linfócitos (3 imagems), e a segunda, com os neutrófilos (10 imagens).

\begin{table}[h]
\centering
\begin{minipage}{0.45\textwidth}
\centering
\begin{tabular}{|c|c|}
\hline
\multicolumn{2}{|c|}{Linfócitos} \\
\hline
linfocito00.png & 3 \\
linfocito01.png & 2 \\
linfocito02.png & 4 \\
\hline
\end{tabular}
\end{minipage}
\hfill
\begin{minipage}{0.45\textwidth}
\centering
\begin{tabular}{|c|c|}
\hline
\multicolumn{2}{|c|}{Neutrófilos} \\
\hline
neutrofilo00.png & 4 \\
neutrofilo01.png & 3 \\
neutrofilo02.png & 2 \\
neutrofilo03.png & 4 \\
neutrofilo04.png & 3 \\
neutrofilo05.png & 2 \\
neutrofilo06.png & 4 \\
neutrofilo07.png & 3 \\
neutrofilo08.png & 2 \\
neutrofilo09.png & 4 \\
\hline
\end{tabular}
\end{minipage}
\end{table}

\section{Isolamento das células}

Inicialmente, importamos as bibliotecas necessárias, nomeadamento o OpenCV, o NumPy e o Matplotlib, responsáveis, respectivamente, pelas operações de visão computacional, manipulação numérica e visualização gráfica.

Considerando que as imagens têm fundo escuro e o citoplasma e núcleos das células estão em tons de púrpura, aplicamos um limiar para converter o fundo branco em preto, isolando a célula no processo. Como o citplasma é mais claro que os núcleos, pode-se aplicar um segundo limiar para separá-los sem que o fundo atrapalhe o processo. O resultado desta etapa pode ser visto abaixo.

\end{document}
